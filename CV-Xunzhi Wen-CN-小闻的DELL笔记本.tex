% !TEX TS-program = xelatex
% !TEX encoding = UTF-8 Unicode
% !Mode:: "TeX:UTF-8"

\documentclass{resume}
\usepackage{zh_CN-Adobefonts_external} % Simplified Chinese Support using external fonts (./fonts/zh_CN-Adobe/)
% \usepackage{NotoSansSC_external}
% \usepackage{NotoSerifCJKsc_external}
% \usepackage{zh_CN-Adobefonts_internal} % Simplified Chinese Support using system fonts
\usepackage{linespacing_fix} % disable extra space before next section
\usepackage{cite}
\usepackage{graphicx}
\usepackage{tabu}
\usepackage{multirow}
\usepackage{progressbar}
\begin{document}
\pagenumbering{gobble} % suppress displaying page number
\begin{table}
  \centering
  \Large{
  \begin{tabu}{ c }
    \centering
    \textbf{\huge{闻逊之}}   \\
    \email{wenxunzhi1@163.com} | \phone{(+86) 173-0552-7206} \\
  \end{tabu}
}
\end{table}

\vspace{2pt}
\section{教育背景}
\datedsubsection{\textbf{西安交通大学} \texorpdfstring{\quad} GPA:3.72/4.0}{西安}
\textit{在读本科生} \quad 信息与计算科学 \hfill{2019.08 -- 至今}

相关课程:抽象代数,偏微分方程,泛函分析,实变函数,数理统计,数值分析,复变函数

\textbf{获奖情况}
\begin{itemize}
  \item \textit{一等奖},腾飞杯创新创业大赛(56/572)\hfill 2021
  \item \textit{荣誉提名},美国大学生数学建模竞赛 (30\%)\hfill 2021,2022
  \item \textit{三等校级奖学金} (10/37) \hfill 2020,2021
  \item \textit{优秀志愿者},中国模拟联合国大会(10/150) \hfill 2019
\end{itemize}

\section{科研经历}
\datedsubsection{\textbf{预期诱导的亚稳态活动建模}}{西安}
\role{研究助理}{指导教师:黄子罡教授,博士生导师,生命科学与技术学院\hfill 2021.06 -- 2022.05}

\begin{itemize}
  \item 将平均场理论用于描述有输入下模型运行时网络行为的全局状态和势能状态及相关参数值
  \item 定义了序列熵、决策敏感度、延迟解码时间等一系列模型评价指标
  \item 调整了模型参数,并在不同刺激输入下获得了清晰的亚稳态图像
  \item 使用多种刺激的组合测试模型,并破译了稳态图像含义
\end{itemize}

\datedsubsection{\textbf{农村农业合作社实地调查与政策分析}}{西安、蚌埠}
\role{课题组长}{指导教师:李聪教授,博士生导师,西安交通大学经济与金融学院\hfill 2020.06 -- 2021.04}
\begin{itemize}
  \item 针对问题背景进行了充分的调研
  \item 在不同省份、不同地区组织并实施了大范围实地调研
  \item 撰写了数据分析与政策建议,最终获得一等奖(56/572)
\end{itemize}

\section{工作经历}
\datedsubsection{\textbf{}}{}

% Reference Test
%\datedsubsection{\textbf{Paper Title\cite{zaharia2012resilient}}}{May. 2015}
%An xxx optimized for xxx\cite{verma2015large}
%\begin{itemize}
%  \item main contribution
%\end{itemize}

\section{IT 技能}
% increase linespacing [parsep=0.5ex]
\begin{itemize}[parsep=0.5ex]
  \item 编程语言: C == Python > C++ > Java
  \item 平台: Linux
  \item 开发: xxx
\end{itemize}


\section{其他}
% increase linespacing [parsep=0.5ex]
\begin{itemize}[parsep=0.5ex]
  \item 技术博客: http://blog.yours.me
  \item GitHub: https://github.com/username
  \item 语言: 英语 - 熟练(TOEFL xxx)
\end{itemize}

%% Reference
%\newpage
%\bibliographystyle{IEEETran}
%\bibliography{mycite}
\end{document}