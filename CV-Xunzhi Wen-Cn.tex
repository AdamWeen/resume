% !TEX TS-program = xelatex
% !TEX encoding = UTF-8 Unicode
% !Mode:: "TeX:UTF-8"

\documentclass{resume}
\usepackage{zh_CN-Adobefonts_external} % Simplified Chinese Support using external fonts (./fonts/zh_CN-Adobe/)
% \usepackage{NotoSansSC_external}
% \usepackage{NotoSerifCJKsc_external}
% \usepackage{zh_CN-Adobefonts_internal} % Simplified Chinese Support using system fonts
\usepackage{linespacing_fix} % disable extra space before next section
\usepackage{cite}
\usepackage{graphicx}
\usepackage{tabu}
\usepackage{multirow}
\usepackage{progressbar}
\begin{document}
\pagenumbering{gobble} % suppress displaying page number
\begin{table}
  \centering
  \begin{tabu}{ c }
    \centering
    \textbf{\huge{闻逊之}}   \\
    \email{wenxunzhi1@163.com} | \phone{(+86) 173-0552-7206} \\
  \end{tabu}
\end{table}

\vspace{2pt}
\section{教育背景}
\datedsubsection{\textbf{西安交通大学} \texorpdfstring{\quad}   GGPA:3.72/4.0}{西安}
\textit{在读本科生} \quad 信息与计算科学 \hfill{2019.08 -- 至今}

相关课程:机器学习,抽象代数,偏微分方程,泛函分析,实变函数,数理统计,数值分析,复变函数

\textbf{获奖情况}
\begin{itemize}
  \item \textit{一等奖},腾飞杯创新创业大赛(56/572)\hfill 2021
  \item \textit{荣誉提名},美国大学生数学建模竞赛 (30\%)\hfill 2021,2022
  \item \textit{三等校级奖学金} (10/37) \hfill 2020,2021
  \item \textit{优秀志愿者},中国模拟联合国大会(10/150) \hfill 2019
\end{itemize}

\section{科研经历}

\datedsubsection{\textbf{时延小样本SLA技术探索}}{西安}
\role{研究助理}{指导教师:\textit{孟德宇教授,博士生导师,西安数学与数学技术研究院}\hfill 2022.04 -- 至今}

\begin{itemize}
  \item 提出基于数据特征的JS散度以及Wassertein距离的数据划分和拼接模型
  \item 制定基于时延异常值的时延可靠性 SLA 评估方法,重新设计服务等级的评价方式
  \item 改进并应用时序卷积网络(TCN),Transformer,Informer等机器学习模型预测异常状态
  \item 各模型训练集上准确率达到95\%以上,测试集上准确率最高达到72.44\%
\end{itemize}

\datedsubsection{\textbf{预期诱导的亚稳态活动建模}}{西安}
\role{研究助理}{指导教师:\textit{黄子罡教授,博士生导师,生命科学与技术学院}\hfill 2021.06 -- 2022.05}

\begin{itemize}
  \item 将平均场理论用于描述有输入下模型运行时网络行为的全局状态和势能状态及相关参数值
  \item 定义了序列熵、决策敏感度、延迟解码时间等一系列模型评价指标
  \item 调整了模型参数,并在不同刺激输入下获得了清晰的亚稳态图像
  \item 使用多种刺激的组合测试模型,并破译了稳态图像含义
  \item 最终定级为省级项目,答辩结果为良好
\end{itemize}

% \datedsubsection{\textbf{农村农业合作社实地调查与政策分析}}{陕西、安徽}
% \role{课题组长}{指导教师:\textit{李聪教授,博士生导师,西安交通大学经济与金融学院}\hfill 2020.06 -- 2021.04}
% \begin{itemize}
%   \item 针对问题背景进行了充分的调研
%   \item 在不同省份、不同地区组织并实施了大范围实地调研
%   \item 撰写了数据分析与政策建议,获得一等奖(56/572)
% \end{itemize}

\section{工作经历}
\datedsubsection{\textbf{深圳依时货拉拉科技有限公司}}{深圳}
\role{司机管理部实习生}{主管: \textit{尹兵兵,运力中心副总裁}\hfill 2021.06 -- 2021.09} 
\begin{itemize}
  \item 协助设计司机会员制度与福利系统
  \item 建立了司机的奖惩机制以规范司机行为
  \item 设计了基于信用积分的司机福利制度
  \item 参与了多场线下司机面谈会,收集并整理司机对会员与福利制度的意见
\end{itemize}


\section{领导力经历}

\datedsubsection{\textbf{模拟联合国大会}}{安徽}
\role{主席团}{省级会议 \hfill 2017.04 -- 2019.09}

\begin{itemize}
  \item 设计并撰写了国际事务与全球治理会议的背景文件
  \item 主持超过50名代表出席的会议,把握会议走向,维持会场秩序
  \item 审查并批准代表撰写的决议草案,为代表颁发奖状
\end{itemize}

\section{技能}
\begin{itemize}[parsep=0.5ex]
  \item 编程语言: C, Python, C\# ,MATLAB, \LaTeX, Microsoft Office
  \item 语言: IELTS 8.0
  \item 爱好: 游泳,萨克斯,羽毛球 %最后不够再加上 
\end{itemize}

\end{document}